\documentclass{beamer}

\title{Lebesgue Integration}
\author{Christian Soltermann}
\date{\today}

\begin{document}
    \begin{frame}
        \titlepage
    \end{frame}
\section{Riemann Integration}
\begin{frame}{Quick Review of Riemann Integration}
    \begin{definition}
        A function $f: \mathbb{R} \rightarrow \mathbb{R}$ is \textbf{Riemann integrable} on an interval $[a,b]$ if \[
            U(f) = L(f) 
        \]
        where $U$ and $L$ are defined as follows
        \begin{gather*}
            U(f) = \inf \{U(f,P) \mid \text{$P$ is a partition of $[a,b]$}\}\\
            L(f) = \sup \{L(f,P) \mid \text{$P$ is a partition of $[a,b]$}\}\\
            U(f,P) = \sum_{i=1}^{n} M_i \Delta x_i \quad M_i = \sup \{f(x) \mid x \in [x_{i-1}, x_i]\}\\
            L(f,P) = \sum_{i=1}^{n} m_i \Delta x_i \quad m_i = \inf \{f(x) \mid x \in [x_{i-1}, x_i]\}
        \end{gather*}
    \end{definition}
\end{frame}

\begin{frame}{Example Riemann Integrable Function}
    \centering
    \includegraphics<1>[width=8cm]{squared0.png}%
    \includegraphics<2>[width=8cm]{squared5.png}%
    \includegraphics<3>[width=8cm]{squared10.png}%
    \includegraphics<4>[width=8cm]{squared20.png}%
    \includegraphics<5>[width=8cm]{squared40.png}%
\end{frame}    

\begin{frame}{Example Non-Riemann Integrable Function}
    \centering
    \includegraphics<1>[width=8cm]{dirichlet0.png}%
    \includegraphics<2>[width=8cm]{dirichlet5.png}%
    \includegraphics<3>[width=8cm]{dirichlet10.png}%
    \includegraphics<4>[width=8cm]{dirichlet20.png}%
    \includegraphics<5>[width=8cm]{dirichlet40.png}%
\end{frame}

\begin{frame}{A Closer Look}
    Riemann integrability can be thought of as a function's ability
    to be approximated by \textbf{step functions}. A step function
    is a function that splits the domain into disjoint intervals
    (a partition) and 
    assigns a single value to each one.\\~\\
    Given a partition $\{x_0, x_1, \ldots\}$ and a set of values $\{a_1, a_2, \ldots\}$,

    \[
        S(x) = a_i\quad\text{if $x \in [x_{i-1}, x_i]$}
    \]

    After defining an integral for step functions, you can
    then extend the definition to any function that can be approximated
    by step functions.

\end{frame}
\begin{frame}{Indicator Functions and Simple Functions}

    \textbf{Lebesgue integration} captures a broader class of functions
    by defining integration not just on step functions, but on 
    \textbf{simple functions}. Simple functions are linear combinations
    of \textbf{indicator functions}, which are defined as 

    \[
        \chi_S(x) = \begin{cases}
            1\quad\text{if $x \in S$}\\
            0\quad\text{otherwise}
        \end{cases}  
    \]
    
    Defining an integral on all simple functions will
    require the machinery of \textbf{measure theory}.
\end{frame}

\section{Measure Theory}

\begin{frame}{$\sigma$-Algebras}
    Given a non-empty set $X$, a \textbf{$\sigma$-algebra} on $X$ is a collection 
    of subsets closed under complements and countable unions. \\~\\
    For any metric space $X$, 
    such as $\mathbb{R}$ with $d(x,y) = |x-y|$, the 
    \textbf{Borel $\sigma$-algebra} is the smallest $\sigma$-algebra
    that includes all open sets (and closed sets by complement) and is denoted $\mathcal{B}_X$.
    \begin{example}
        For the set $X = \{a,b,c,d\}$, $\Sigma = \{\emptyset, \{a,b\}, \{c,d\}, \{a,b,c,d\}\}$
        is a $\sigma$-algebra. 
    \end{example}
\end{frame}

\begin{frame}{Measures}

    A measure is an function $\mu$ from a $\sigma$-algebra to 
    the extended-real interval $[0,\infty]$ that satisfies some simple properties.
    \begin{enumerate}[(i)]
        \item $\mu(\emptyset) = 0$
        \item For a countable collection of disjoint sets $\{S_i\}_{i \in I}$
        \[\mu\left(\bigcup_{i\in I} S_i\right) = \sum_{i\in I} \mu(S_i)\]
    \end{enumerate}

    A subset with measure 0 is called a \textbf{null set}.
    \\~\\
    A space $X$, $\sigma$-algebra $\mathcal{M}$, and 
    measure $\mu$ form a \textbf{measure space}
    $(X, \mathcal{M}, \mu)$.

\end{frame}

\begin{frame}{The Lebesgue Measure}

    Using $\mathbb{R}$ as a metric space and taking the Borel
    $\sigma$-algebra $\mathcal{B}_\mathbb{R}$, a measure can
    be defined
    \[
       m(S) := \inf\left\{\sum_{j=1}^{\infty}\left(b_j-a_j\right) : \bigcup_{j=1}^{\infty}(a_j,b_j] \supset S\right\} 
    \] 
    In words, the measure of an interval is defined to be its length. Then, an arbitrary set can
    be covered by a collection of half-open intervals. The measure of an 
    arbitrary set is defined as the infimum of the summed lengths of all possible collections.
    \\~\\
    After completion of $\mathcal{B}_\mathbb{R}$ and $m$
    (a process that assigns measure zero to all subsets of null sets),
    the completed $\sigma$-algebra $\mathcal{L}$ and measure $\mu$ define
    the \textbf{Lebesgue measure}. 
    \\~\\
    "Measurable" is used to mean Lebesgue-measurable ($\in \mathcal{L}$).

\end{frame}

\begin{frame}{Example: Measure of $\mathbb{Q}$}
Recall that $\mathcal{B}_\mathbb{R}$ includes all closed sets in $\mathbb{R}$ and that
$\mathbb{Q}$ is a countable union of closed (singleton) sets.
Therefore, $\mathbb{Q} \in \mathcal{B}_\mathbb{R} \subset \mathcal{L}$
and so $\mathbb{Q}$ is Lebesgue measurable. 
\\~\\
Each point $x_n \in \mathbb{Q}$ can be placed in a half-interval of length 
$\frac{\epsilon}{2^n}$. Then,
\[
    \mu(\mathbb{Q}) \leq \sum_{n=0}^{\infty} \frac{\epsilon}{2^n} = \epsilon
\]
Since $\epsilon$ is arbitrarily small, $\mu(\mathbb{Q}) = 0$. 
\end{frame}

\begin{frame}{Example: Non-Measurability of Vitali Sets}
Recall the equivalence relation from the midterm, where 
$x~y \iff x - y \in \mathbb{Q}$. This equivalence class
breaks the real line into shifted copies of the rationals.
A Vitali set $N$ includes one representative from each equivalence class within the interval
$[0,1)$.
\\~\\
It can be shown that the interval $[0,1)$ is equal to a countable
union of shifted copies of $N$, and so
\[
    1 = \mu([0,1)) = \sum_{0}^{\infty} m(N)
\]
\\~\\
If $m(N) = 0$, then $1=0$. If $m(N) > 0$, then $1=\infty$.
In either case, a contradiction. Therefore $N$ cannot be
measurable.

\end{frame}

\begin{frame}{Measurable Functions}
    Recall that one characterization of continuous states that a function
    $f$ is continuous if the preimage $f^{-1}(S)$ of open any open set $S$
    is open.
    \\~\\
    Measurable functions can be defined in a similar fashion. A function $f$
    is said to be measurable if the preimage of any set in $\mathcal{B}_\mathbb{R}$
    is (Lebesgue) measurable.
    \\~\\
    Measurable functions are closed under the usual operations: addition,
    subtraction, multiplication, and division. In addition, they are closed 
    limit operations. So if a sequence of measurable functions converge, their
    limit is also a measurable function.
\end{frame}

\begin{frame}{Defining the Lebesgue Integral}
    For some simple function $\phi = \sum_{i=1}^{n} a_i \chi_{S_i}$ (where
    $S_i$ are all disjoint), its \textbf{Lebesgue integral} is defined 
    \[
        \int_{\mathbb{R}}\phi\,d\mu := \sum_{i=1}^{n} a_i\mu(S_i).
    \]
    Then, for a non-negative Lebesgue measurable function $f_+$, the integral
    is defined
    \[
        \int_{\mathbb{R}}f_+\,d\mu := \sup \left\{\int_{\mathbb{R}} \phi\,d\mu : \phi \leq f_+, f_+ \text{ is simple}\right\}.
    \]
\end{frame}

\begin{frame}{The Full Lebesgue Integral (Finally)}
    Finally, the integral of an arbitrary Lebesgue measurable function $f$ can be defined
    by splitting the function into its positive and negative parts, $f = f_+ - f_-$ (the two
    parts can be shown to be Lebesgue measurable).
    \[
        \int_{\mathbb{R}}f\,d\mu := \int_{\mathbb{R}}f_+\,d\mu - \int_{\mathbb{R}}f_-\,d\mu  
    \]
    If both the positive and negative integrals are finite, $f$ is said to be \textbf{Lebesgue integrable}.
    Equivalently, a function is Lebesgue integrable if
    \[
        \int_{\mathbb{R}} |f|\,d\mu < \infty.  
    \]
\end{frame}

\begin{frame}{Partitioning the Range}
    Lebesgue integration works by approximating a function $f$ with simple
    functions. A sequence of approximations can be created by partitioning
    the range of $f$.
    \\~\\
    Given a partition of the range, $P = \{y_0, y_1, y_2, \ldots, y_n\}$,
    let $E_k = f^{-1}((y_{k-1}, y_k])$. Then, 
    \[
      s(x) = \sum_{k=1}^{n} y_k \chi_{E_k}  
    \]
    is a simple function approximating $f$. A sequence of approximations
    can be created using finer and finer partitions.
\end{frame}

% \begin{frame}{Example Lebesgue Integral}
%     \centering
%     \includegraphics<1>[width=8cm]{lsquared0.png}%
%     \includegraphics<2>[width=8cm]{lsquared5.png}%
%     \includegraphics<3>[width=8cm]{lsquared10.png}%
%     \includegraphics<4>[width=8cm]{lsquared20.png}%
%     \includegraphics<5>[width=8cm]{lsquared40.png}%
% \end{frame}

\begin{frame}{Properties of the Lebesgue Integral}
    The Lebesgue integral is an extension of the Riemann integral,
    meaning that any function that is Riemann integral is also
    Lebesgue integral and the two integrals agree.
    \\~\\
    The Lebesgue integral also maintains all the nice properties
    of the Riemann integral, such as linearity and the fundamental
    theorem of calculus.
    \\~\\
    One difference between the Riemann and Lebesgue integrals is that 
    the Lebesgue integral does not specify orientation. For the Riemann
    integral, $\int_a^b f\,dx = -\int_b^a f\,dx$. For the Lebesgue
    integral, $\int_{[a,b]} f\,d\mu$ has no concept of direction.  
\end{frame}

\begin{frame}{What's the Point?}
    The Lebesgue integral not only captures a larger space of 
    functions than the Riemann integral, but also defines
    a nicer space. The Lebesgue integral, unlike the Riemann
    integral, behaves well under limits.
    \begin{theorem}[Monotone Convergence Theorem]
        If $\{f_n : n \in \mathbb{N}\}$ is a monotone increasing sequence of positive,
        measurable, extended real-values functions $f_n : X \rightarrow [0,\infty]$ and 
        $f = \lim_{n\to\infty} f_n$, then
        \[
            \lim_{n\to\infty} \int f_n\, d\mu = \int f\,d\mu   
        \]
    \end{theorem}
    Note that this theorem does not hold for Riemann integration.
\end{frame}

\begin{frame}{One Last Theorem}
    \begin{theorem}[Dominated Convergence Theorem]
        If $\{f_n\}$ is a sequence of measurable functions
        $f_n : X \to \mathbb{R}$ such that $f_n \to f$ pointwise,
        and $|f_n| \leq g$ where $g : X \to [0, \infty]$ is an 
        integrable function, then
        \[
            \lim_{n\to\infty}\int f_n\,d\mu = \int f\,d\mu      
        \]
    \end{theorem}

    These limit theorems are very powerful tools for all fields of mathematics
    that deal with sequences of functions, especially probability theory and
    Fourier analysis.
\end{frame}
\end{document}